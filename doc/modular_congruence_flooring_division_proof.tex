%This proof is used in Emboss to propagate alignment restrictions through
%the flooring division operator "//": if a and b are integers and b is a
%constant, then a // b can still have a known alignment.
%
%Forgive me on the details and formatting, math majors, I never completed my
%math degree.  --bolms
\documentclass{article}
\usepackage[utf8]{inputenc}
\usepackage[english]{babel}
\usepackage{amssymb,amsmath,amsthm}
\title{Modular Congruence of Flooring Division of a Value with Known Modular Congruence by a Constant}
\author{Emboss Authors}
\date{2023}

\begin{document}
\maketitle

\newtheorem{theorem}{Theorem}

%(Draft)

\begin{theorem}
Given

\begin{align*}
  &{a} \equiv {r} \pmod{{m}} \\
  &{b} \neq 0 \\
  &{a}, {r}, {m}, {b}, {s}, {n} \in \mathbb{Z} \\
  &\dfrac{{m}}{{b}} \in \mathbb{Z}
\end{align*}

then

\begin{align}
  &\left\lfloor\dfrac{{a}}{{b}}\right\rfloor \equiv \left\lfloor\dfrac{{r}}{{b}}\right\rfloor \pmod{\dfrac{{m}}{{b}}} \label{eq:result}
\end{align}
\end{theorem}

\begin{proof}
By the definition of modular congruence:
\begin{align}
  &\exists {x} \in \mathbb{Z} : {a} = {m}{x} + {r} \label{eq:aexpand} \\
  &\left\lfloor\dfrac{{a}}{{b}}\right\rfloor \equiv \left\lfloor\dfrac{{r}}{{b}}\right\rfloor \pmod{\dfrac{{m}}{{b}}} \Leftrightarrow \exists {z} \in \mathbb{Z} : \left\lfloor\dfrac{{a}}{{b}}\right\rfloor = \dfrac{{m}}{{b}}{z} + \left\lfloor\dfrac{{r}}{{b}}\right\rfloor \\
  &\left\lfloor\dfrac{{a}}{{b}}\right\rfloor \equiv \left\lfloor\dfrac{{r}}{{b}}\right\rfloor \pmod{\dfrac{{m}}{{b}}} \Leftrightarrow \exists {x}, {z} \in \mathbb{Z} : \left\lfloor\dfrac{{m}{x} + {r}}{{b}}\right\rfloor = \dfrac{{m}}{{b}}{z} + \left\lfloor\dfrac{{r}}{{b}}\right\rfloor \label{eq:resultexpand}
\end{align}

given that $\dfrac{{m}}{{b}} \in \mathbb{Z}$ and that the product of two integers is an integer:
\begin{align}
  \dfrac{{m}}{{b}}{x} \in \mathbb{Z} \label{eq:mxbint}
\end{align}

by the known property of \textit{floor}:
\begin{align}
  \forall {n} \in \mathbb{Z} : \lfloor{c} + {n}\rfloor = \lfloor{c}\rfloor + {n}
\end{align}

and \eqref{eq:mxbint}:
\begin{align}
  \left\lfloor\dfrac{{m}{x} + {r}}{{b}}\right\rfloor
    &= \left\lfloor\dfrac{{m}{x}}{{b}} + \dfrac{{r}}{{b}}\right\rfloor \\
    &= \left\lfloor\dfrac{{m}}{{b}}{x} + \dfrac{{r}}{{b}}\right\rfloor \\
    &= \dfrac{{m}}{{b}}{x} + \left\lfloor\dfrac{{r}}{{b}}\right\rfloor \label{eq:floorintextract}
\end{align}

substituting \eqref{eq:floorintextract} into \eqref{eq:resultexpand}:
\begin{align}
  &\left\lfloor\dfrac{{a}}{{b}}\right\rfloor \equiv \left\lfloor\dfrac{{r}}{{b}}\right\rfloor \pmod{\dfrac{{m}}{{b}}} \Leftrightarrow \exists {x}, {z} \in \mathbb{Z} : \dfrac{{m}}{{b}}{x} + \left\lfloor\dfrac{{r}}{{b}}\right\rfloor = \dfrac{{m}}{{b}}{z} + \left\lfloor\dfrac{{r}}{{b}}\right\rfloor
\end{align}

choosing an arbitrary integer $n$ for $x$ and $z$ specializes the left equation:
\begin{align}
  \dfrac{{m}}{{b}}{n} + \left\lfloor\dfrac{{r}}{{b}}\right\rfloor = \dfrac{{m}}{{b}}{n} + \left\lfloor\dfrac{{r}}{{b}}\right\rfloor \Rightarrow
  \exists {x}, {z} \in \mathbb{Z} : \dfrac{{m}}{{b}}{x} + \left\lfloor\dfrac{{r}}{{b}}\right\rfloor = \dfrac{{m}}{{b}}{z} + \left\lfloor\dfrac{{r}}{{b}}\right\rfloor \label{eq:tautology}
\end{align}

since the equation on the left of \eqref{eq:tautology} is a tautology, the
equation on the right of \eqref{eq:tautology} is true.  By
\eqref{eq:resultexpand}, this implies that \eqref{eq:result} is true.

\end{proof}
\end{document}
